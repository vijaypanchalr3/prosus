This is our master’s semester four project report. We have done work on instrumentation for finding magnetic susceptibility of unknown samples. Our final goal is to find phase transitions of samples under change of temperature. We got needed help from the department of physics’s condensed matter laboratory, Dr. Utpal Joshi, Swati Pachauri. Lock-In amplifier that we got access from Utpal sir and it  was a necessary component in our project. 

In this project we did first to design and improvise our instrument, which is an AC susceptometer. Major problem faced by such a setup by non-expert is the problem of noise. We gave our best time for this to get as minimum noise as possible. This was a very big task for us. Another thing which made us think is that of getting offset null for AC susceptometer. We did this by changing the coil's parameter and relative orientation of the coils with each other. Final giant in our project was calibration, as it turns out our instrumentation has some noise floor and for samples, especially that of paramagnetic, have very feeble magnetization which is hard to detect with our setup. So, we choose ferromagnetic samples such as nickel, LSMO and $Gd_2O_3$. Problem with these samples was that their magnetic susceptibility is highly dependent on conditions of measurement, so we can’t find its absolute values like paramagnetic samples. This problem seems to be big for calibration of our instrument. Finally we have tried and successfully calibrated with some accuracy. Final goal was to find temperature dependent magnetic susceptibility plots and find curie temperature where magnetic phase transition happens. This task does not need very accurate calibration and is possible with our instrumentation.
