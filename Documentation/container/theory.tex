%% \subsection{Magnetism}
We should know little about some concepts of magnetism. Magnetism in loops is profoundly due to relative motions of charge particles. This is related to concepts of relativity (especially Einstein’s special relativity). As we know moving charge creates a magnetic field which curls around movement direction. These magnetic fields get accumulated with a number of loops which relate to our instrumentations. Also, changing velocity (relates directly to changing current) creates a changing magnetic field.

There’s another type of magnetism which differs in some minor differences. This is related to the material’s magnetic moments. As we know atoms have magnetic moments related with two momentums,orbital angular momentum and spin angular momentum. This degrees of freedom of electrons (as by product relates to atoms) creates magnetic properties of materials. 

%% \subsection{Spins}
First evident by Stern and Gerlach 

A magnetic susceptibility of a material can be defined as the amount of a material gets magnetised , when it is placed in an external magnetic field .
In other words , an amount of magnetization of a material occurs when it is placed in external magnetic field .

An expression for magnetic susceptibility is given by ,

                  Χ = M / H

Where , M= Magnetic susceptibility 
              H= Applied/External magnetic field 

There are two other measures of susceptibility:

1. MAGNETIC MOLAR SUSCEPTIBILITY (Χm)

2. MASS MAGNETIC SUSCEPTIBILITY (Χр)

                   Χm = M Χр

Where,  Χр= Χv / р , р= density (kg/m^3)

Magnetic susceptibility is a factor that indicates the magnetic behavior of a material.It gives an idea about a material that it can be attracted or can be repelled. 


CLASSIFICATION OF MATERIALS BASED ON THEIR MAGNETIC PROPERTIES :-

DIAMAGNETIC MATERIAL :

A magnetic materials which aligns its domains or field lines against the applied magnetic field are known as diamagnetic materials.

These materials are strongly repelled by the magnets .

As these materials gets magnetize in opposite side of applied field , they have a small amount of magnetization .

Example :- water, tin , mercury ,etc.

They have magnetic susceptibility Χ < 0,
negative value of magnetic 
susceptibility.

          2 .  PARAMAGNETIC MATERIAL :

A magnetic materials which aligns its domains or field lines with the applied field are known as paramagnetic materials .

These materials are weakly attracted by the magnets and also they are temperature dependent .

Example :- aluminium , alkaline earth metals , etc.

They have magnetic susceptibility Χ > 0 , positive value of magnetic 
susceptibility.

            3 .  FERROMAGNETIC MATERIAL :

A magnetic materials that are highly gets magnetized in an external magnetic field are known as ferromagnetic materials .

These materials are highly attracted by the magnets .

Example :- iron, cobalt, nickel , etc .

They have magnetic susceptibility Χ > 1, always higher value of magnetic susceptibility.
 
There are further classification of materials based on their magnetic properties can be done as :

a) Anti-ferromagnetic materials 

b) Ferrimagnetic materials 

These two types are not discussed in detail because they are not in context to our project work .

WHAT DO YOU MEAN BY AN AC MAGNETIC SUSCEPTIBILITY ?

AC magnetic measurements are taken by applying AC field to the samples and resulting AC magnetic moment is measured i.e.,
induced by changing magnetic flux by applied AC field .

This results in the different values of magnetic susceptibility for different values of magnetic flux arises from the different values of AC field.

In order to understand AC magnetic susceptibility , first we have consider measurements at low frequencies , where the 
measurements is almost equals to the DC susceptibility.In this case the absolute value of magnetization is calculated i.e.,

 Χ = M / H

In case of AC susceptibility, the continuously varying value of magnetization , hence varying value of susceptibility i.e.,

 Χac = (dM / dH)

AC susceptibility is often referred as dynamic susceptibility.AC measurements are very sensitive to the small changes in the values of magnetization .

